\documentclass[a4paper, 11pt,oneside]{article}
\usepackage[
  top=1.5cm,
  bottom=1cm,
  left=2cm,
  right=1.5cm,
  headheight=25.22153pt, % as per the warning by fancyhdr
  includehead,includefoot,
  heightrounded, % to avoid spurious underfull messages
]{geometry} 

\usepackage[T1]{fontenc}
\usepackage{microtype}
\usepackage{fancyhdr}
\usepackage{fancyvrb}
\usepackage{lipsum}
\usepackage{url}
\usepackage{listings}
\usepackage{lastpage}
\usepackage{enumitem}
\usepackage{datetime}

\settimeformat{hhmmsstime}
\yyyymmdddate

\pagestyle{fancy}
\fancyhf{} % clear all fields

\pagestyle{fancy}
\lhead{CMSC 125: Operating Systems \\ First Semester 2020-2021}
\rhead{Institute of Computer Science \\ University of the Philippines Los Banos}
\rfoot{JACHermocilla [CC-BY-NC-SA 4.0]}
%\cfoot{Enjoy!:)}
\cfoot{\thepage\ of \pageref{LastPage}}
\lfoot{Revision: \today\ \currenttime}
%\rfoot{https://jachermocilla.org/teaching/125}
\renewcommand{\headrulewidth}{0.4pt}
\renewcommand{\footrulewidth}{0.4pt}

\begin{document}

\begin{center}
	{\LARGE \textbf{ICS-OS Lab 01: Building ICS-OS}}
\end{center}

\section*{Objectives}
   At the end of this activity, you should be able to:
   \begin{enumerate}[itemsep=0pt,parsep=0pt]
       \item build the ICS-OS kernel binary image;
       \item build the ICS-OS floppy disk image; and
       \item boot ICS-OS and run two commands.
   \end{enumerate}   

\section{Introduction}
ICS-OS is an instructional operating system that can be used for understanding different operating system concepts.  An operating system is no different from other software in that it is written in a programming language, such as C. 

The build process creates the compressed kernel binary image (\texttt{vmdex}) and the floppy disk image (\texttt{ics-os-floppy.img}). Since ICS-OS has a relatively large number of source files, 
the \texttt{Make} utility is used for the build. You can examine the contents of \texttt{Makefile} 
to learn more of the details how this is done.

To start ICS-OS, the floppy disk image is loaded in Qemu and is set as the boot device. It contains 
the GRUB bootloader which transfers control to the ICS-OS kernel (in \texttt{vmdex}). After the boot process, a prompt is provided to users to enter commands. 

\section{Prerequisites}
The recommended development environment is Ubuntu 16.04. Tasks described here may or may not 
work on other Linux distributions. Familiarity with the command line is also needeed.

\section{Deliverables and Credit}
Perform the tasks below and capture screen shots while you do them. Submit a PDF file 
containing the screen shots with captions. Do not forget to put your name and laboratory section.
Credit is five (5) points.


\section{Tasks}

\subsection*{Task 1: Install build dependencies}
A C compiler, assembler, linker, emulator, bootloader and other tools are needed to 
build and run ICS-OS. Run the following commands to install the dependencies.

\begin{lstlisting}[language=bash,frame=single]
$sudo apt-get update
$sudo apt-get install build-essential nasm qemu-kvm tcc git gcc-multilib \
      grub-pc-bin xorriso
\end{lstlisting}

\subsection*{Task 2: Clone the repository and explore the source tree}
ICS-OS\footnote{https://github.com/srg-ics-uplb/ics-os/} is open source and is hosted on Github. Run the following command to 
checkout the source code and explore the source tree.
\begin{lstlisting}[language=bash,frame=single] 
$git clone https://github.com/srg-ics-uplb/ics-os.git
$cd ics-os/ics-os
\end{lstlisting}

\subsection*{Task 3: Build}
Run the commands below to build. The commands should be repeated 
whenever changes in the kernel source code are made. You will modify the kernel source 
code in future homeworks. 
\begin{lstlisting}[language=bash,frame=single] 
$make clean
$make vmdex
$make floppy 
\end{lstlisting}

\subsection*{Task 4: Boot}
Start Qemu with the floppy image as boot device using the command below. 
\begin{lstlisting}[language=bash,frame=single] 
$make boot-floppy 
\end{lstlisting}

\subsection*{Task 5: Run ICS-OS commands}
Once the ICS-OS command prompt (\texttt{\%}) appears, type \texttt{help}. 
Examine the list of commands and run two of these commands. Do not forget 
to capture screen shots of the outputs.

%\begin{thebibliography}{9}
%\end{thebibliography}

\end{document}