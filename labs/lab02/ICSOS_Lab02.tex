\documentclass[a4paper, 11pt,oneside]{article}
\usepackage[
  top=1.5cm,
  bottom=1cm,
  left=2cm,
  right=1.5cm,
  headheight=25.22153pt, % as per the warning by fancyhdr
  includehead,includefoot,
  heightrounded, % to avoid spurious underfull messages
]{geometry} 

\usepackage[T1]{fontenc}
\usepackage{microtype}
\usepackage{fancyhdr}
\usepackage{fancyvrb}
\usepackage{lipsum}
\usepackage{url}
\usepackage{listings}
\usepackage{lastpage}
\usepackage{enumitem}
\usepackage{datetime}
\usepackage{minted}

\settimeformat{hhmmsstime}
\yyyymmdddate

\pagestyle{fancy}
\fancyhf{} % clear all fields

\pagestyle{fancy}
\lhead{CMSC 125: Operating Systems \\ First Semester 2020-2021}
\rhead{Institute of Computer Science \\ University of the Philippines Los Banos}
\rfoot{JACHermocilla | CC BY-NC-SA 4.0}
%\cfoot{Enjoy!:)}
\cfoot{\thepage\ of \pageref{LastPage}}
\lfoot{Revision: \today\ \currenttime}
%\rfoot{https://jachermocilla.org/teaching/125}
\renewcommand{\headrulewidth}{0.4pt}
\renewcommand{\footrulewidth}{0.4pt}

\begin{document}

\begin{center}
   {\LARGE \textbf{ICS-OS Lab 02: Command Line Interface, System Calls, and System
Utilities }}
\end{center}

\section*{Objectives}
   At the end of this activity, you should be able to:
   \begin{enumerate}[itemsep=0pt,parsep=0pt]
       \item add a a new console command;
       \item add a new system call service/function; and
       \item invoke a system call from a system utility.
   \end{enumerate}   

\section{Introduction}
The command line interface (CLI) is one way an operating system allows users to
access its services such as program execution.  The user enters a command
string on the CLI prompt and the OS executes the command. The execution usually
involves \textit{system calls} which invoke some services provided by the 
kernel. This services are functions that execute in \textit{previliged or 
kernel mode}. The CLI can be implemented as part of the kernel itself, as in 
the case of ICS-OS, or as a separate program, called a \textit{shell}, which is 
outside of the kernel.  Since shells (and other system utilities and user 
applications as well) are not in the kernel but may require system calls, they 
do so using software interrupts (32-bit Linux uses \texttt{int 80h}, DOS uses 
\texttt{int 21h}, and ICS-OS uses \texttt{int 30h}). Application Programming 
Interfaces(APIs), Software Development Kit(SDKs), and Runtime Environments 
(REs) make it easy to write programs for operating systems by hiding the 
details of the systems calls from the programmers.


\section{Prerequisites}
To proceed with this lab, you should have completed \texttt{Lab 01}. Most of 
the commands that we will use in this lab will be run relative to the 
\texttt{\$ICSOS\_HOME/ics-os} directory. Update your local copy of the source 
code and create a new branch for this lab with the commands below.

\begin{minted}[frame=single,framesep=10pt]{bash}
$cd $ICSOS_HOME/ics-os
$git checkout master
$git pull
$git checkout -b lab02
$git branch    #to check the currrent branch
\end{minted}


\section{Deliverables and Credit}
Perform the tasks below and capture screenshots while you do them. Answer
all questions. Submit a PDF file containing the screen shots with captions
and answers to questions. Do not forget to put your name and laboratory
section.  Credit is ten (10) points.

\section{Tasks}

\subsection*{Task 1: Add a new console command (3 points)} 
The CLI in ICS-OS is part of the kernel. Its implementation is located in
\texttt{kernel/console/console.c}. The function \texttt{int 
console\_execute(const char *str)} is where the command string (what you type
in the \texttt{\%} prompt) is processed. Study this function. The 
\texttt{strtok()} function is used to tokenize the command string to extract
the command name and its arguments. The code fragment below is for the new
\texttt{add} command with two integer arguments that we wish to include.
Insert the code fragment in an appropriate location in the 
\texttt{console\_execute()} function. Build and boot ICS-OS (as discussed in 
Tasks 3-5 in Lab 01) to test if the command works. Capture screenshots. Also 
show where you placed the code fragment. 

\begin{minted}[frame=single,framesep=10pt]{C}
if (strcmp(u,"add") == 0){   //-- Adds two integers. Args: <num1> <num2> 
   int a, b; 
   u = strtok(0," "); 
   a = atoi(u);
   u = strtok(0," "); 
   b = atoi(u); 
   printf("%d + %d = %d\n",a,b,a+b); 
}else
\end{minted}

\textbf{QUESTION}: What are the advantages and disadvantages of having the CLI 
in the kernel?

\subsection*{Task 2: Add a new system call service/function (3 points)}
The list of functions/services accessible through system calls are placed in a
\textit{system call table}. In ICS-OS, it is the array of structures 
defined in \texttt{kernel/dexapi/dex32API.h}: 

\begin{minted}[frame=single,framesep=10pt]{C}
api_systemcall api_syscalltable[API_MAXSYSCALLS];
\end{minted}

The function \texttt{api\_init()} in \texttt{kernel/dexapi/dex32API.c}
populates this table. 

ICS-OS hooks to \texttt{int 30h} to handle system calls. This is set in
\texttt{kernel/hardware/chips/irqhandlers.c}. Recall the interrupt system 
and the interrupt vector table (IVT) discussed in the lecture.

\begin{minted}[frame=single,framesep=10pt]{C}
setinterruptvector(0x30, dex_idtbase, 0xEE, syscallwrapper, SYS_CODE_SEL); 
\end{minted}

You do not have to understand all the parameters of the above function for now. 
The parameter \texttt{syscallwrapper} is a function defined in 
\texttt{kernel/irqwrap.asm}. It calls the function \texttt{api\_syscall(...)} 
from \texttt{kernel/dexapi/dex32API.c} which processes the system call and 
invokes the appropriate service from the system call table. 
\texttt{api\_syscall(...)} is called everytime \texttt{int 30h} is generated.

\begin{minted}[frame=single,framesep=10pt]{C}
DWORD api_syscall(DWORD fxn,DWORD val,DWORD val2,
                  DWORD val3,DWORD val4,DWORD val5);
\end{minted}


To add a system call, the \texttt{api\_addsystemcall()} function is used. Its 
prototype is shown below.
\begin{minted}[frame=single,framesep=10pt]{C}
int api_addsystemcall(DWORD function_number, void *function_ptr,
                        DWORD access_check, DWORD flags);
\end{minted}

The important parameters are \texttt{function\_number} and 
\texttt{function\_ptr}. Say you want to implement the 
\texttt{kchown()}\footnote{Usually functions in the kernel are written with the 
letter \texttt{k} at the start} system call function/service below that changes 
the owner of a file. Edit \texttt{kernel/dexapi/dex32API.c} and add the 
function.

\begin{minted}[frame=single,framesep=10pt]{C}
int kchown(int fd, int uid, int gid){
   printf("Changing owner of fd=%d to user id=%d and group id=%d\n", fd, uid, gid);
   //Actual code to change file ownership is placed here.   
   return 0; //0-success
}
\end{minted}

To add it to the system call table, add the following line in the 
\texttt{api\_init()} function. The function number you will use is 
\texttt{0xC2}. Take note of this number.

\begin{minted}[frame=single,framesep=10pt]{C}
api_addsystemcall(0xC2, kchown, 0, 0);
\end{minted}

Capture screenshots where you placed the codes. 
Build ICS-OS. At this point, the new system call is added to the kernel but it is not 
doing anything yet. You will do that in Task 3. 

\subsection*{Task 3: Invoke a system call in a system utility (4 points)}
In this task you are to make a system utility that invokes the system call service  
you created in Task 2. In ICS-OS, system utilities and user applications are 
placed in the \texttt{contrib} folder. There is an example application, 
\texttt{hello}, which you will use as template. Study the \texttt{Makefile}. 
Run the commands below to create your \texttt{chown} system utility.  

\begin{Verbatim}[frame=single]
$ cd contrib	   #go to the contrib folder
$ cp -r hello/ chown/   #copy hello to chown
$ cd chown/             #go inside chown
$ mv hello.c chown.c    #rename hello.c to chown .c
$ sed -i 's/hello/chown/g' Makefile   #replace hello with chown in the Makefile
$ make
$ make install
\end{Verbatim}

Go back to \texttt{\$ICSOS\_HOME}. Build and boot ICS-OS. 
Inside ICS-OS, run the following commands and capture screenshots.

\begin{Verbatim}[frame=single]
% cd apps
% ls -l -oname
% chown.exe
\end{Verbatim}

\textbf{QUESTION}: What is the output of executing \texttt{chown.exe} inside ICS-OS?

Go back to the \texttt{contrib/chown} folder. Edit \texttt{chown.c} and replace 
the contents with the code below. 

\begin{Verbatim}[frame=single]
#include "../../sdk/dexsdk.h"
int main(int argc, char *argv[]) {
   if (argc < 4){ 
      printf("Usage: chown.exe <fd> <uid> <gid>\n");
      return -1; 
   }   
   dexsdk_systemcall(0x9F, atoi(argv[1]), atoi(argv[2]), 
                     atoi(argv[3]), 0, 0); 
   return 0;
}
\end{Verbatim}

\textbf{QUESTION}: Study the function \texttt{dexsdk\_systemcall()} defined in \texttt{sdk/tccsdk.c}.
What does this function do?

Now build and install the \texttt{chown} system utility.
\begin{Verbatim}[frame=single]
$ make
$ make install
\end{Verbatim}

Go back to \texttt{\$ICSOS\_HOME}. Build and boot ICS-OS. 
Inside ICS-OS, run the following commands and capture screenshots.

\begin{Verbatim}[frame=single]
% cd apps
% ls -l -oname
% chown.exe
\end{Verbatim}

\textbf{QUESTION}: What is the output of executing \texttt{chown.exe} this time?


%\begin{thebibliography}{9} \end{thebibliography}
\section{Tips}
You can use the \texttt{grep} utility to quickly search for strings in files from \texttt{\$ICSOS\_HOME} .
\begin{Verbatim}[frame=single]
$ grep -rn api_init
\end{Verbatim}
 

\end{document}
